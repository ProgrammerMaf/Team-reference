
\documentclass[14pt,a4paper,oneside]{article}
% \documentclass[14pt,a4paper,oneside,twocolumn]{article}

\renewcommand{\normalsize}{\large}

\usepackage[utf8]{inputenc}
\usepackage[russian]{babel}
\usepackage[left=2cm,right=2cm,top=2cm,bottom=2cm,bindingoffset=0cm]{geometry}
\usepackage{listings}
\usepackage{fancyhdr}
\usepackage{anyfontsize}

\lstset{
  	language=C++,
%	aboveskip=3mm,
% 	belowskip=3mm,
%  	showstringspaces=false,
	identifierstyle=,
	keywordstyle=\bfseries,
  	columns=flexible,
  	tabsize=4,
 	basicstyle=\ttfamily,
% 	breaklines=true,
%  	breakatwhitespace=true,
}

%\pagenumbering{arabic}

\headsep=10mm          % отступ после колонтитула (10mm)

\makeatletter%ставим, чтобы знак @ воспринимался как буква, а не как команда

%\renewcommand{\@oddhead}{\LARGE\textbf{Ural Federal University}\hfil\LARGE\thepage}%верхний колонтитул для нечетных страниц
%\renewcommand{\@evenhead}{\LARGE\textbf Ural Federal University\hfil\LARGE\thepage}%верхний колонтитул для нечетных страниц

\fancypagestyle{plain}{%
    \fancyhf{}
    \renewcommand{\headrulewidth}{0.0pt}
    \renewcommand{\footrulewidth}{0.0pt}
    \lhead{Ural Federal University}
    \chead{Ural FU: Strong Statement}
    \rhead{WF 2019, page \thepage\ of\ 25}
    \cfoot{}
}
\pagestyle{plain}


\makeatother

\begin{document}

\section{Configure files}

\lstinputlisting{sources/config.cpp}

\newpage

Перед контестом:
1) Посмотреть, что используется: stdin/stdout или файлы (если файлы, то какие у них имена?)
2) Проверить директиву ONLINE\_JUDGE
3) Понять, как печатать
4) long double, \_\_int128
5) assert, throw~--- какие вердикты?
6) Проверить, что есть и работают студия и питон
7) Сделать файл с "JUDGEID".

Перед задачей:
1) Проверить руками сэмплы
2) Подумать, как дебагать после написания
3) Выписать сложные формулы и все $\pm 1$

Перед сабмитом:
1) Файлы
2) Output-debug в том числе stderr
3) Потестить на семплах
4) Если python: обернуть всё в main
5) Flush в интерактивных задачах
6) Переполнения int
7) Размеры массивов
8) Вбить псевдо-максимальный тест
9) cout.precision

Если случился WA:
1) Проверить на переполнения int и long long
2) Перечитать условие
3) Перечитать код
4) Рассказать решение другому члену команды
5) Написать стресс/интерактор/чекер
6) Проверить, что в функциях не забыт return
7) Потестить не только ответ, но и содержимое промежуточных переменных
8) Изменить тесты так, чтобы ответ не менялся: пошафлить, поменять размер блока, корень дерева, растянуть координаты
9) Поставить assert
10) Проверить, что программа не печатает что-то неожиданное: inf, NaN, неправильное количество чисел, пустой ответ, не лекс. мин. решение

Если случился TL:
1) cin -> scanf -> getchar
2) Сгенерить max-тест
3) STL -> свои реализации
4) Попробовать уместить массивы в кеш. Поменять местами for-ы и/или измерения массивов.
5) Проверить, что из-за debug-output мы не совершаем лишних действий (не обходим всё дерево, чтобы вывести его).

Если случился RE:
1) Выход за границы вектора
2) Stack overflow
3) Выход за область определения (arcsin(500))

\subsection{Прагмы}
\lstinputlisting[language=C++]{sources/pragmas.cpp}

\subsection{Аллокатор}
\lstinputlisting[language=C++]{sources/allocator.cpp}

\section{Математика}

\subsection{Условие существования генератора}
Первообразные корни существуют только по модулям $m$ вида $2, 4, p^a, 2p^a$, где $p$ -- простое число, $p > 2$.

~

~

\subsection{Простые числа}
$10^4 + 7, 10^4 + 9, 10^6 + 3, 998244353$

$10^9 + 7, 10^9 + 9, 10^9 + 21, 10^9 + 33, 10^9 + 123$

$10^{15} + 159, 10^{18} + 3, 10^{18} + 31, 10^{18} + 3111$

\subsection{Highly composite numbers}
$d(840) = 32, d(9240) = 64$

$d(83160) = 128, d(720720) = 240$

$d(8648640) = 448, d(91891800) = 768$

$d(931170240) = 1344$

$d(97772875200) = 4032$

$d(963761198400) = 6720$

$d(866421317361600) = 26880$

$d(897612484786617600) = 103680$

\subsection{Интегралы}

\[
\int \frac{\,dx}{a^2 + x^2} = \frac{1}{a}arctan\frac{x}{a}, a \ne 0
\]

\[
\int \frac{\,dx}{a^2 - x^2} = \frac{1}{2a}ln\left|\frac{a+x}{a-x}\right|, a \ne 0
\]
  
\[
\int \frac{x\,dx}{a^2 \pm x^2} = \pm\frac{1}{2}ln|a^2 \pm x^2|
\]

\[
\int \frac{\,dx}{\sqrt{a^2 - x^2}} = arcsin\frac{x}{a}, a > 0
\]
  
\[
\int \frac{\,dx}{\sqrt{x^2 \pm a^2}} = ln\left|x + \sqrt{x^2 \pm a^2}\right|, a > 0
\]

\[
\int \frac{x\,dx} {\sqrt{a^2 \pm x^2}} = \pm \sqrt{a^2 \pm x^2} , a > 0
\]

\[
\int {\sqrt{a^2 - x^2}}\,dx = \frac x 2 \sqrt{a^2 - x^2} + \frac {a^2} 2 \arcsin{\frac x a}, a > 0
\]

\[
\int {\sqrt{x^2 \pm a^2}}\,dx = \frac x 2 \sqrt{x^2 \pm a^2} \pm \frac {a^2} 2 \ln{|x + \sqrt{x^2 \pm a^2}|}, a > 0
\]

\[
\int {\sqrt{\frac{a + x}{b + x}}}\,dx = \sqrt{(a + x)(b + x)} + (a - b)\ln{(\sqrt{a + x} + \sqrt{b + x})}
\]

\[
\int {\sqrt{\frac{a - x}{b + x}}}\,dx = \sqrt{(a - x)(b + x)} + (a + b)\arcsin{\sqrt{\frac{x + b}{a + b}}}
\]

\[
\int {\sqrt{\frac{a + x}{b - x}}}\,dx = -\sqrt{(a + x)(b - x)} - (a + b)\arcsin{\sqrt{\frac{b - x}{a + b}}}
\]

\subsection{Матан}

\hspace*{1.3em}Длина плоской кривой $y = f(x)$

\[
l = \int\limits_{x_1}^{x_2} \sqrt{1 + f’(x)^2} \,dx = \int\limits_{t_1}^{t_2} \sqrt{x_t'^{~2} + y_t'^{~2}} \,dt
\]

Площадь поверхности вращения кривой $y = f(x)$ вокргу оси $OX$:

\[
S = 2\pi \int\limits_{x_1}^{x_2} f(x)\sqrt{1 + f’(x)^2} \,dx
\]

Объем поверхности вращения кривой $y = f(x)$ вокргу оси $OX$:

\[
V = \pi \int\limits_{x_1}^{x_2} f(x)^2 \,dx
\]

Формула Симпсона

\[
\int\limits_a^b f(x) \,dx = {\frac h 3}\left({\frac{f(x_0) + f(x_n)} 2} + \sum_{i=1}^{n-1} f(x_i) + 2 \sum_{i=1}^{n-1} f\Bigl(x_{i+ \frac 1 2}\Bigr) \right)
\]

\subsection{Треугольник}

\[
m_a = \frac 1 2 \sqrt{2(b^2+c^2) - a^2}
\]

\[
l_a = \frac {2\sqrt{bcp(p-a)}} {b + c}
\]

\subsection{Сфера}

\hspace*{1.3em}Площадь сегмента сферы: $S=2\pi R h$

Объем сегмента сферы: $V=\pi h^2 \left(R - \frac h 3\right)$

Объем сектора сферы: $V=\frac 2 3 \pi h^2 R$

Расстояние между точками на сфере:

$L = R \arccos(\cos{\theta_1} \cos{\theta_2} + $

$\sin{\theta_1} \sin{\theta_2} \cos(\phi_1 - \phi_2))$

где $\theta$~--- широты (от $-{\frac \pi 2}$ до ${\frac \pi 2}$)

$\phi$~--- долготы (от $-\pi$ до $\pi$)

\subsection{Лемма Бернсайда}
Чтобы посчитать классы эквивалентности, нужно для каждой валидной перестановки посчитать количество неподвижных 
точек (объекты которым пофиг на перестановку), сложить все это и поделить на количество перестановок

$ClassesCount = {\frac 1 {|G|}} \sum_{\pi \in G} I(\pi)$

\subsection{Хакенбуш}
$G$ -- функция Гранди.
$G(T) = \oplus (G(T_i) + 1)$

Сделаем
сначала первое наблюдение: петля в графе с точки зрения игры эквивалентна
висячему ребру, исходящему из этой вершины.

Опишем теперь метод как любой граф с циклами можно превратить в
дерево, которое имеет такую же функцию Гранди. Рассмотрим две вершины,
лежащие на цикле. Объединим их в одну, превратив ребра, соединявшие эти
вершины, в петли. Оказывается, получившийся граф будет иметь такую же
функцию Гранди, как и исходный граф. Продолжая процесс пока в графе не
исчезнут все циклы, получим в результате дерево.

\subsection{Количество способов сделать граф связным}

$s_1 \cdot s_2 \cdot \ldots \cdot s_k \cdot n^{k-2}$

Где $n$~--- кол-во вершин, $k$~--- кол-во компонент,
$s_i$~--- размеры компонент

\section{Алгоритмы}

%\subsection{Суффиксный массив}
%\lstinputlisting[language=C++]{sources/suffixArray.cpp}

%\subsection{Массив циклических сдвигов}
%\lstinputlisting[language=C++]{sources/shiftArray.cpp}

%\subsection{Быстрое преобразование Фурье}
%\lstinputlisting[language=C++]{sources/fft.cpp}

%\subsection{Алгоритм ДД}
%\lstinputlisting[language=C++]{sources/dinic.cpp}

\subsection{Алгоритм удаления циклов отрицательного веса}
Если каждый раз искать любой отрицательный цикл, то экспоненциальное время работы. Если каждый раз искать цикл с минимальной средней стоимостью (что также можно производить за $O(nm)$), то время работы всего алгоритма уменьшится до $O(nm^2 \log n)$, однако никто не знает, есть ли тест, где достигается.

\subsection{Формула из дейкстры с потенциалами}
\lstinputlisting[language=C++]{sources/formula_from_dijkstra_with_potentials.cpp}

 
\subsection{Алгоритм Эдмондса}
\lstinputlisting[language=C++]{sources/edmond.cpp}

%\subsection{Префикс-функция}
%\lstinputlisting[language=C++]{sources/prefix.cpp}

\subsection{Суффиксный автомат}%Олег
\lstinputlisting[language=C++]{sources/automaton.cpp}

%\subsection{Ахо-Корасик}%Ероп
%\lstinputlisting[language=C++]{sources/krasic.cpp}

\subsection{Декомпозиция Линдона}
\lstinputlisting[language=C++]{sources/lyndon_decomposition.cpp}

%\subsection{Компоненты двусвязности}%Олег
%\lstinputlisting[language=C++]{sources/ts.cpp}

\subsection{Иосиф Флавий}
\lstinputlisting[language=C++]{sources/joseph.cpp}

\subsection{Теория чисел}
\lstinputlisting[language=C++]{sources/number_theory.cpp}

\subsection{Пересечение полуплоскостей}
\lstinputlisting[language=C++]{sources/halfPlaneInter.cpp}

\subsection{Берлекемп-Месси}
\lstinputlisting[language=C++]{sources/bm.cpp}

\subsection{Венгерка}
\lstinputlisting[language=C++]{sources/hungarian.cpp}

\subsection{Укконен}
\lstinputlisting[language=C++]{sources/ukkonen.cpp}

\subsection{Суф мас}
\lstinputlisting[language=C++]{sources/suf_array_log2.cpp}

\subsection{Касаи}
\lstinputlisting[language=C++]{sources/kasai_from_neerc_wiki.cpp}

\subsection{Персистентное Декартово дерево}
\lstinputlisting[language=C++]{sources/Treap.cpp}

\newpage
\section{Запускаем Данилючка, работяги}
{\fontsize{4}{4.8}\selectfont
\lstinputlisting[language=C++]{sources/daniluk.txt}
}

\newpage

\begin{tabular}{|c|c|c|c|c|}
\hline
&comment&Валя&Саша&Коля\\
\hline
\begin{minipage}{1cm}
~\\
\centering
A
~\\
~\\
\end{minipage}
&
\begin{minipage}{7,5cm}
~\\
~\\
~\\
~\\
\end{minipage}
&
\begin{minipage}{2,5cm}
~\\
~\\
~\\
\end{minipage}
&
\begin{minipage}{2,5cm}
~\\
~\\
~\\
\end{minipage}
&
\begin{minipage}{2,5cm}
~\\
~\\
~\\
\end{minipage}
\\
\hline
\begin{minipage}{1cm}
~\\
~\\
\centering
B
~\\
~\\
~\\
\end{minipage}
&&&&\\
\hline
\begin{minipage}{1cm}
~\\
~\\
\centering
C
~\\
~\\
~\\
\end{minipage}
&&&&\\
\hline
\begin{minipage}{1cm}
~\\
~\\
\centering
D
~\\
~\\
~\\
\end{minipage}
&&&&\\
\hline
\begin{minipage}{1cm}
~\\
~\\
\centering
E
~\\
~\\
~\\
\end{minipage}
&&&&\\
\hline
\begin{minipage}{1cm}
~\\
~\\
\centering
F
~\\
~\\
~\\
\end{minipage}
&&&&\\
\hline
\begin{minipage}{1cm}
~\\
~\\
\centering
G
~\\
~\\
~\\
\end{minipage}
&&&&\\
\hline
\begin{minipage}{1cm}
~\\
~\\
\centering
H
~\\
~\\
~\\
\end{minipage}
&&&&\\
\hline
\begin{minipage}{1cm}
~\\
~\\
\centering
I
~\\
~\\
~\\
\end{minipage}
&&&&\\
\hline
\begin{minipage}{1cm}
~\\
~\\
\centering
J
~\\
~\\
~\\
\end{minipage}
&&&&\\
\hline
\begin{minipage}{1cm}
~\\
~\\
\centering
K
~\\
~\\
~\\
\end{minipage}
&&&&\\
\hline
\begin{minipage}{1cm}
~\\
~\\
\centering
L
~\\
~\\
~\\
\end{minipage}
&&&&\\
\hline
\begin{minipage}{1cm}
~\\
~\\
\centering
M
~\\
~\\
~\\
\end{minipage}
&&&&\\
\hline
\end{tabular}
\newpage

\end{document}
                              
